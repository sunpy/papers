\documentclass[12pt]{iopart}

\usepackage{natbib}
\usepackage{minted}
\definecolor{bg}{rgb}{0.95, 0.95, 0.95}
\usepackage{hyperref}
%Uncomment next line if AMS fonts required
%\usepackage{iopams}
\usepackage{graphicx}
\graphicspath{{./images/}}
\newcommand*{\newblock}{Bibliography}
\begin{document}

\title{SunPy - Python for Solar Physics}

\author{The SunPy Community}
\address{\url{http://sunpy.org}}
\ead{sunpy@googlegroups.com}

\author{Stuart J Mumford}
\address{Solar Physics \& Space Plasma Research Centre (SP$^{2}$RC), School of 
Mathematics and Statistics, The University of Sheffield, Hicks Building, 
Hounsfield Road, Sheffield, S3 7RH U.K.}

\author{Steven Christe}
\address{NASA Goddard Space Flight Center, Greenbelt, U.S.A.}

\author{David P\'erez-Su\'arez}
\address{South African National Space Agency - Space Science,
Hospital Street, 7200 Hermanus, Western Cape, South Africa}

\author{Jack Ireland}
\address{ADNET Systems Inc., Mail Code 671.1, NASA Goddard Space
  Flight Center, Greenbelt, MD, U.S.A.}

\begin{abstract}
This paper presents version 0.4 of SunPy a community developed Python package 
for solar physics.

\end{abstract}

\maketitle

\section{Introduction}\label{sec:Intro}

Science is driven by the analysis of data of ever-growing variety and 
complexity. Advances in sensor technology, combined with the availability of 
inexpensive 
storage, have led to rapid increases in the amount of data available to scientists in almost
every discipline.  Solar physics is no exception to this trend. For example,
NASA's \textit{Solar Dynamics Observatory} (\textit{SDO}) spacecraft, launched
in February 2010, produces over 1 TB of data per day \citep{pesnell2012}. Managing and
analysing these data requires increasingly sophisticated software
tools. These tools should be robust, easy to use and modify, have a transparent
development history, and conform to modern software-engineering
standards. Software with these qualities provide a strong foundation that can support the
needs of the community as data volumes grow and science questions evolve.

%The SunPy project aims to provide a free, open-source, and openly developed
%software package for the analysis and visualisation of solar data.
The SunPy project aims to provide a software package with these qualities for 
the analysis and visualisation of solar data. SunPy makes
use of Python and scientific Python packages. Python is a free, general-purpose, 
powerful, and easy-to-learn high-level programming language. Additionally, Python is 
widely used outside of scientific fields in areas such as `big data' analytics, web 
development, and educational environments. For example, \texttt{pandas} \citep{mckinney2010, mckinney2012} was 
originally developed for quantitative analysis of financial data and has since 
grown into a generalised time-series data-analysis package. Python continues to 
see increased use in the astronomy community \citep{greenfield2011}, which has 
similar goals and requirements as the solar physics community. Finally, Python 
integrates well with many technologies such as web servers \citep{dolgert2008} and databases. 

The development of a package such as SunPy is made possible by the rich ecosystem of 
scientific packages available in Python. Core packages such as \texttt{NumPy}, 
\texttt{SciPy} \citep{jones2001}, and \texttt{matplotlib} \citep{hunter2007} provide 
the basic functionality expected of a scientific programming language,
such as array manipulation, core numerical algorithms, and visualisation, respectively.
Building upon these foundations, packages such as \texttt{astropy}
\citep[astronomy;][]{theastropycollaboration2013}, \texttt{pandas} (time-series), and \texttt{scikit-image} 
\citep[image processing;][]{vanderwalt2014} provide more domain-specific functionality.

A typical workflow begins with a solar physicist manually identifying
a small number of events of interest on the Sun.  This is typically
done in order to investigate in detail the physics of these events
(for example, the large solar flare of 23 July 2002 has Astrophysical
Journal Letters volume 595, dedicated to its analysis).
In this workflow, an event is investigated in depth which requires 
data from many different instruments.
These data are typically provided in many different formats - for
example, FITS \citep[Flexible Image Transport System,][]{refId0}, CSV, or
binary files - and contain many different types of data (such as
images, lightcurves and spectra).  In addition, the repositories these data reside
in can have different access methods.  This workflow is characterized
by the large number of heterogeneous datasets used in the
investigation of a small number of solar events.

Another typical workflow begins with the solar physicist identifying a
large sample of data or events.  The goal here is obtain information
about the population in general.  An example might be to calculate the
fractal dimension of a large number of active region magnetic fields
\citep{2005ApJ...631..628M}, or to calculate the observed temperatures
in a population of solar flares \citep{2012ApJS..202...11R}.  This
workflow is typically characterized by lower data heterogeneity, but
with a larger number of files.

The volume and variety of solar data used in these workflows drives
the need for an environment in which obtaining and performing common
solar physics operations on these data is as simple and intuitive as
possible.  SunPy is designed to be a clean, simple-to-use, and
well-structured open-source package that provides the \textit{core}
tools for solar data analysis, motivated by the need for a free and
modern alternative to the existing SolarSoft (SSW) library
\citep{freeland1998}. While SSW is open source and freely available,
it relies on IDL (Interactive Data Language), a proprietary
data-analysis environment.

The purpose of this paper is to provide an overview of SunPy's current
capabilities, an overview of the project's development model, community aspects of the
project, and future plans. The latest release of SunPy, version 0.5,
can be downloaded from \url{http://sunpy.org} or can be
installed using the Python package index (\url{http://pypi.python.org/pypi}).

\label{sec:Intro}

\section{Core Data Types}\label{sec:DataTypes}

The core of SunPy is a set of data structures that are specifically
designed for the three primary varieties of solar physics data:
images, time series, and spectra. These core data types are supported
by the SunPy classes: \texttt{Map} (2D spatial data),
\texttt{LightCurve} (1D temporal series), and \texttt{Spectrum} and
\texttt{Spectrogram} (1D and 2D spectra).  The purpose of these
classes is to provide the same core data type to the SunPy user
regardless of the differences in source data.  For example, if two
different instruments use different time formats to describe the
observation time of their images, the corresponding SunPy \texttt{Map}
object for each of them expresses the observation time in the same
way.  This simplifies the workflow for the user when handling data
from multiple sources. 

These classes allow access to the data
and associated metadata and provide appropriate convenience functions to
enable analysis and visualisation. For each of these classes, the data is
stored in the \texttt{data} attribute, while the metadata is stored 
in the \texttt{meta} attribute\footnote{Note, that currently only \texttt{Map} and \texttt{LightCurve} have this feature 
fully implemented}. 
It is possible to instantiate the
data types from various
different sources: e.g., files, URLs, and arrays.  
In order to provide instrument-specific specialisation, the core SunPy classes 
make use of subclassing; e.g., \texttt{Map} has an \texttt{AIAMap} 
sub-type for data from the \textit{SDO}/AIA (Atmospheric Imaging Assembly; \citealt{lemen2012}) instrument. 

All of the core SunPy data types 
include visualisation methods that are tailored to each data type. 
These visualisation methods all utilise the \texttt{matplotlib} 
package and are designed in such a way that they integrate well with 
the \texttt{pyplot} functional interface of \texttt{matplotlib}.

This design philosophy makes the behaviour of SunPy's visualisation 
routines intuitive to those who already understand the \texttt{matplotlib}
interface, as well as allowing the use of the standard 
\texttt{matplotlib} commands to manipulate the plot parameters (e.g., title, axes).
Data visualisation is provided by two functions: \texttt{peek()}, for quick 
plotting, and \texttt{plot()}, for plotting with more fine-grained control.

This section will give a brief overview of the \textit{current} functionality 
of each of the core SunPy data types.

\subsection{Map}\label{ssec:map}
The map data type stores 2D spatial data, such as images of the Sun and 
inner heliosphere. It provides: a wrapper around a \texttt{numpy} data array, 
the images associated spatial coordinates, and other metadata. The \texttt{Map} 
class provides methods for typical operations on 2D data, such as rotation and 
re-sampling, as well as visualisation.
The \texttt{Map} class also provides a convenient interface for loading data 
from a variety of sources, including from FITS files, the standard format for storing image data in solar physics and astrophysics community. 
An example of creating a \texttt{Map} object from a FITS file is shown in 
Listing~\ref{code:aia_1}.

The architecture of the map subpackage consists of a template map called
\texttt{GenericMap}, which is a subclass of \texttt{astropy.nddata.NDData}. 
\texttt{NDData} is a generic wrapper around a \texttt{numpy.ndarray} with a 
\texttt{meta} attribute to store metadata.
As \texttt{NDData} is currently still in development, \texttt{GenericMap} does 
not yet make full use of its capabilities, but this inheritance structure 
provides for future integration with \texttt{astropy}. In order to provide 
instrument- or detector-specific integration, \texttt{GenericMap} is designed
to be subclassed. Each subclass of \texttt{GenericMap} can register 
with the \texttt{Map} creation factory, which will then automatically return an instance
of the specific \texttt{GenericMap} subclass dependent upon the data provided. 
SunPy v0.5 has \texttt{GenericMap} specialisations for the following 
instruments: 

\begin{itemize}
\item \textit{Yohkoh}/SXT - SXT: Solar X-ray Telescope \citep{1991SoPh..136....1O, 1991SoPh..136...37T},
\item \textit{SOHO}/EIT and LASCO - SOHO: Solar and Heliospheric Observatory \citep{domingo1995}, EIT: Extreme Ultraviolet Telescope \citep{1995SoPh..162..291D} and LASCO: Large Angle Spectroscopic COronagraph \citep{1995SoPh..162..357B}
\item \textit{RHESSI} - RHESSI: Reuven Ramaty High Energy Solar Spectroscopic Imager \citep{2002SoPh..210....3L},
\item \textit{STEREO}/EUVI and COR1/2 - STEREO: Solar TErrestrial RElations Observatory \citep{2005AdSpR..36.1483K}, EUVI: Extreme Ultraviolet Imager \citep{2004SPIE.5171..111W} and COR1/2: CORonagraph 1/2 \citep{2002AdSpR..29.2017H}
\item \textit{Hinode}/XRT - XRT: X-Ray Telescope \citep{2007SoPh..243....3K, 2007SoPh..243...63G}.
\item \textit{PROBA2}/SWAP - PROBA2: PRojects for On Board Autonomy 2 \citep{2013SoPh..286....5S}, SWAP: Sun Watcher Active Pixel \citep{2013SoPh..286...43S}
\item \textit{SDO}/AIA and HMI - HMI: Helioseismic Magnetic Imager, \citep{2012SoPh..275..207S}
\item \textit{IRIS} - IRIS: Interface Region Imaging Spectrograph \citep{2011SPD....42.1512L} SJI (slit-jaw imager) frames.           
\end{itemize}
             
The \texttt{GenericMap} class stores all of the metadata retrieved from the header of
the image file in the \texttt{meta} attribute and provides convenience 
properties for commonly accessed metadata: e.g., \texttt{instrument}, 
\texttt{wavelength} or \texttt{coordinate\_system}.
These properties are dynamic mappings to the underlying metadata and all methods 
of the \textit{GenericMap} class modify the meta data where needed.
For example, if \verb|aiamap.meta[`instrume']| is modified then \verb|aiamap.instrument| 
will reflect this change.
Currently this is implemented by not preserving the keywords of the input data,
instead modifying meta data to a set of ``standard" keys supported by SunPy.
Listing \ref{code:aia_1} demonstrates the quick-look functionality of 
\texttt{Map}.

\begin{listing}[H]
\pythoncode{pycode_map1.txt}
\begin{center}
\includegraphics[width=0.8\columnwidth]{aia_map_example}
\end{center}
\caption{Example of the \texttt{AIAMap} specialisation of 
\texttt{GenericMap}. First, a map is created from a sample \textit{SDO}/AIA FITS file. In this case, a demonstration file contained within the SunPy repository is used. A cutout
of the full map is then created by specifying the desired solar-$x$ and solar-$y$ ranges of the plot in data coordinates (in this case, arcseconds), and then a quick-view plot is created with lines of heliographic longitude and latitude over-plotted.}
\label{code:aia_1}
\end{listing}

In addition to the data-type classes, the \texttt{map} subpackage provides two 
collection classes, \texttt{CompositeMap} and \texttt{MapCube}, for 
spatially and temporally aligned data respectively.
\texttt{CompositeMap} provides methods for overlaying spatially aligned 
data, with support for visualisation of images and contour lines overlaid 
upon each other.
\texttt{MapCube} provides methods for animation of its series of \texttt{Map} 
objects. Listings~\ref{code:compmap_1} and \ref{code:mapcube_1} show how to 
interact with these classes.

\begin{listing}[H]
\pythoncode{pycode_map2.txt}
\begin{center}
\includegraphics[width=0.8\columnwidth]{comp_map_example}
\end{center}
\caption{Example showing the functionality of \texttt{CompositeMap}, with RHESSI X-ray image data composited
on top of an \textit{SDO}/AIA 1600 $\AA$ image. The \texttt{CompositeMap} is plotted using the integration with the \texttt{matplotlib.pyplot} interface.}
\label{code:compmap_1}
\end{listing}

\begin{listing}[H]
\pythoncode{pycode_map3.txt}
\begin{center}
\includegraphics[width=0.8\columnwidth]{aia_cube_controls}
\end{center}
\caption{Example showing the creation of a \texttt{MapCube} from a list of AIA image files. The 
resultant plot makes use of \texttt{matplotlib}'s interactive widgets to allow scrolling 
through the \texttt{MapCube}.}
\label{code:mapcube_1}
\end{listing}

\subsection{Lightcurve}\label{ssec:lightcurve}

Time series data and their analyses are a fundamental part of solar
physics for which many data sources are available.
SunPy provides a \texttt{LightCurve} class
with a convenient and consistent interface for handling solar time-series
data.  The main engine behind the \texttt{LightCurve} class is
the {\texttt{pandas}} data analysis library.  
\texttt{LightCurve}'s \texttt{data} attribute is a \texttt{pandas.DataFrame} 
object. The \texttt{pandas} library contains a large amount
of functionality for manipulating and analysing time-series data,
making it an ideal basis for \texttt{LightCurve} \citep{mckinney2012}.  \texttt{LightCurve}
assumes that the input data are time-ordered list(s) of numbers, and each
list becomes a column in the \texttt{pandas} DataFrame object.

Currently, the \texttt{LightCurve} class is compatible with the
following data sources: the Geostationary Operational Environmental
Satellite (\textit{GOES}) X-ray Sensor (XRS), the \textit{Nobeyama
  Radioheliograph (NoRH)}, \textit{PROBA2} Large Yield Radiometer
(LYRA, \citealt{2013SoPh..286...21D}), \textit{RHESSI},
\textit{SDO} EUV Variability Experiment\footnote{Note that only the level ``OCS'' and average
  CSV files is currently implemented -- see
  \url{http://lasp.colorado.edu/home/eve/data/}} (EVE, \citealt{2012SoPh..275..115W}). 
\texttt{LightCurve}
also supports a number of solar summary indices - such as average
sunspot number - that are provided by the National Oceanic and
Atmospheric Administration (NOAA).  For each of these sources, a
subclass of the \texttt{LightCurve} object is initialised (e.g.,
\texttt{GOESLightCurve}) which inherits from \texttt{LightCurve}, but
allows instrument-specific functionality to be included.  Future
developments will introduce support for additional instruments and
data products, as well as implementing an interface similar to that of
\texttt{Map}.  Since there is no established standard as to how
time-series data should be stored and distributed, each SunPy
\texttt{LightCurve} object subclass provides the ability to download
its corresponding specific data format in its constructor and parse
that file type. A more general download interface is currently in development.

A \texttt{LightCurve} object may be created using a number of different methods. 
For example, a \texttt{LightCurve} may be created for a specific instrument given
an input time range. In Listing~\ref{code:goes_lc}, 
the \texttt{LightCurve} constructor searches a remote source for the GOES X-ray 
data specified by the time interval, downloads the required files, and 
subsequently creates and plots the object. Alternatively, if the data file 
already exists on the local system, the \texttt{LightCurve} object may be 
initialised using that file as input.

\begin{listing}[H]
\pythoncode{pycode_lightcurve.txt}
\begin{center}
\includegraphics[width=10cm]{goes_lightcurve.pdf}
\end{center}
\caption{Example retrieval of a GOES lightcurve
using a time range and the output of the 
\texttt{peek()} method. The maximum flux value in the \textit{GOES} 1.0--8.0$\AA$\ channel 
is then retrieved along with the location in time of the maximum.}
\label{code:goes_lc}
\end{listing}

\input{2-3-Spectra}

\label{sec:DataTypes}

\input{3-Retrieval}
\label{sec:retrevial}

\section{Additional functionality}\label{sec:util}
SunPy is meant to provide a consistent environment for solar data analysis. In 
order to achieve this goal SunPy provides a number of additional functions and packages which 
are used by the other SunPy modules and are made available to the user. This section 
briefly describes some of these functions.
	
\subsection{World Coordinate System (WCS) Coordinates}\label{ssec:util:wcs}
Coordinate transformations are frequently a necessary task within the solar 
data analysis workflow. Likely the most often used transformation is from 
observer coordinates (e.g., sky coordinates) to a coordinate system that is 
mapped onto the solar surface (e.g., latitude and longitude). This 
transformation is necessary to compare the true physical distance between 
different solar features. This type of transformation is not unique
to solar observations, but is not often considered by astronomical packages
such as the Astropy 
\texttt{coordinates} package. The \texttt{wcs} package in SunPy implements the World Coordinate 
System (WCS) for solar coordinates as described by \cite{thompson2006}. The 
transformations currently implemented are those most useful 
for most solar data analysis, namely converting from Helioprojective-Cartesian 
(HPC) to Heliographic (HG) coordinates. HPC describes the positions on 
the Sun as angles measured from the center of the solar disk (usually in 
arcseconds) using Cartesian coordinates (X, Y). This is the coordinate system 
most often defined in solar imaging data (see for example, images from 
\textit{SDO}/AIA, \textit{SOHO}/EIT, and \textit{TRACE}). 
HG coordinates express positions on the Sun using longitude and latitude on 
the solar sphere. There are two standards for this coordinate system:
Stonyhurst-Heliographic, where the origin is at the intersection of the solar 
equator and the central meridian as seen from Earth, and 
Carrington-Heliographic, which is fixed to the Sun and does not depend on Earth. The 
implementation of these transformations pass through a common coordinate system 
called Heliocentric-Cartesian (HCC), where positions are expressed in true 
(de-projected) physical distances instead of angles on the celestial sphere.
These transformations require some knowledge of the location of the observer, 
which is usually provided by the image header. In the cases where it is 
not provided, the observer is assumed to be at Earth. Listing \ref{code:wcs_code} shows 
some examples of coordinate transforms carried out in SunPy using the 
\texttt{wcs} utilities. 

\begin{listing}[H]
\begin{minted}[bgcolor=bg]{pycon}
>>> from sunpy import wcs
>>> wcs.convert_hg_hpc(10, 53)
(100.49244115330731, 767.97438321917502)
# Convert that position back to heliographic coordinates
>>> wcs.convert_hpc_hg(100.49, 767.97)
(9.9996521808465175, 52.999563684874893)
# Try to convert a position which is not on the Sun to HG
>>> wcs.convert_hpc_hg(-1500, 0)
sunpy/wcs/wcs.py:180: RuntimeWarning: invalid value encountered in sqrt
  distance = q - np.sqrt(distance)
(nan, nan)
# Convert sky coordinate to a position in HCC
>>> wcs.convert_hpc_hcc(-300, 400, z=True)
(-216716967.63331246, 288956420.9477042, 594364636.2208252)
\end{minted}
\caption{Using the \texttt{wcs} subpackage.}
\label{code:wcs_code}
\end{listing}

\subsection{Sun}\label{ssec:util:sun}
The purpose of the \texttt{sun} subpackage is to provide solar-specific data such as ephemerides and
solar constants. The main namespace contains a number of functions that provide solar
ephemerides such as the Sun-to-Earth distance, solar-cycle number, the mean 
anomaly, etc.
All of these functions take a time as their input, which can be provided in a format
compatible with \texttt{sunpy.time.parse\_time()}. 

The \texttt{sun.constants} module provides a number of solar-related 
constants in order to provide consistency in the calculations of derived solar 
values  within the SunPy code base, but also to the user. Every solar 
constant is provided as a \texttt{Constant} object as defined by Astropy. Each 
\texttt{Constant} object defines a \texttt{Quantity}, a number associated with a unit, along with 
the constant's provenance (i.e., reference) and its uncertainty. Using 
Astropy's \texttt{Quantity} objects, any solar constant can easily be converted between 
different units, including between the SI or cgs unit systems, as can be seen in Listing~\ref{code:constants_code}.
As these objects inherit from 
NumPy's \texttt{ndarray}, they work well with standard representations of numbers.
For convenience, a number of shortcuts to frequently used constants are provided 
directly when importing the module. A larger list of constants can be 
accessed through an interface modelled on that provided by the SciPy constants 
package and is available as a dictionary called \texttt{physical\_constants}. 
To view them all quickly, a \texttt{print\_all()} function is available.

\begin{listing}[H]
\begin{minted}[bgcolor=bg]{pycon}
>>> from sunpy.sun import constants
>>> print(constants.mass)
  Name   = Solar mass
  Value  = 1.9891e+30
  Error  = 5e+25
  Units  = kg
  Reference = Allen's Astrophysical Quantities 4th Ed.
# Verify the average density of the Sun and convert to cgs
>>> (constants.mass/constants.volume).cgs
<Quantity 1.40851154227 g / (cm3)>
# Search for the age of the Sun
>>> constants.find('age')
['age', 'average angular size', 'average density', 'average intensity']
>>> constants.value('age'), constants.unit('age')
(4600000000.0, Unit("yr"))
\end{minted}
\caption{Using the \texttt{sun.constants} module.}
\label{code:constants_code}
\end{listing}
	
\subsection{Instruments}\label{ssec:util:inst}
In addition to providing support for instrument-specific solar data via the main data 
classes \texttt{Map}, \texttt{LightCurve}, and \texttt{Spectrum}, 
some instrument-specific functions may be found within the \texttt{instr} subpackage. 
These functions are generally those that are unique to one particular solar instrument, 
rather than of general use, such as a function to construct a \textit{GOES} flare event list 
or a function to query the \textit{LYRA} timeline annotation file. Currently, some support is included
 for the \textit{GOES}, \textit{LYRA}, \textit{RHESSI} and \textit{IRIS} instruments, while future developments 
 will include support for additional missions. Ultimately, it is anticipated that solar
  missions requiring a large suite of software tools will each be supported via a separately 
  maintained package that is affiliated with SunPy.


\label{sec:util}

\input{5-Dev}
\label{sec:dev}

\input{6-Future}
\label{sec:future}

\section{Conclusion}

%%% References
\bibliographystyle{stunat}
\bibliography{Sunpy_paper_0.4}{}

\end{document}
