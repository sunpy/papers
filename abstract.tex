\begin{abstract}
This paper presents version 0.4 of SunPy, a community-developed Python package 
for solar physics.
Python, a free, cross-platform, general-purpose, high-level programming 
language, has seen widespread adoption among the scientific community, resulting 
in the availability of a large number of software packages, from numerical 
computation (\texttt{NumPy}, \texttt{SciPy}) and machine learning (\texttt{scikit-learn}) to visualisation 
and plotting (\texttt{matplotlib}).
SunPy is a data-analysis environment specialising in providing the software 
necessary to analyse solar and heliospheric datasets in Python. 
SunPy is open-source software (BSD licence) and has an open and transparent 
development workflow that anyone can contribute to.
SunPy provides access to solar data through integration with the Virtual 
Solar Observatory (VSO), the Heliophysics Event Knowledgebase (HEK), and the 
HELiophysics Integrated Observatory (HELIO) webservices. It currently supports image data from major solar missions (e.g., 
\textit{SDO}, \textit{SOHO}, \textit{STEREO}, and \textit{IRIS}), time-series data from 
missions such as \textit{GOES}, \textit{SDO}/EVE, and \textit{PROBA2}/LYRA, and radio 
spectra from e-Callisto and \textit{STEREO}/SWAVES. We describe SunPy's functionality,
provide examples of solar data analysis in SunPy, and show how Python-based solar 
data-analysis can leverage the many existing tools already 
available in Python. We discuss the future goals of the project and encourage 
interested users to become involved in the planning and development of SunPy.
\end{abstract}
