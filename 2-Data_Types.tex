\section{Core Data Types}\label{sec:DataTypes}

The core of SunPy is a set of data structures that are specifically
designed for the three primary varieties of solar physics data:
images, time series, and spectra. These core data types are supported
by the SunPy classes: \texttt{Map} (2D spatial data),
\texttt{LightCurve} (1D temporal series), and \texttt{Spectrum} and
\texttt{Spectrogram} (1D and 2D spectra).  The purpose of these
classes is to provide the same core data type to the SunPy user
regardless of the differences in source data.  For example, if two
different instruments use different time formats to describe the
observation time of their images, the corresponding SunPy \texttt{Map}
object for each of them expresses the observation time in the same
way.  This simplifies the workflow for the user when handling data
from multiple sources. 

These classes allow access to the data
and associated metadata and provide appropriate convenience functions to
enable analysis and visualisation. For each of these classes, the data is
stored in the \texttt{data} attribute, while the metadata is stored 
in the \texttt{meta} attribute\footnote{Note, that currently only \texttt{Map} and \texttt{LightCurve} have this feature 
fully implemented}. 
It is possible to instantiate the
data types from various
different sources: e.g., files, URLs, and arrays.  
In order to provide instrument-specific specialisation, the core SunPy classes 
make use of subclassing; e.g., \texttt{Map} has an \texttt{AIAMap} 
sub-type for data from the \textit{SDO}/AIA (Atmospheric Imaging Assembly; \citealt{lemen2012}) instrument. 

All of the core SunPy data types 
include visualisation methods that are tailored to each data type. 
These visualisation methods all utilise the \texttt{matplotlib} 
package and are designed in such a way that they integrate well with 
the \texttt{pyplot} functional interface of \texttt{matplotlib}.

This design philosophy makes the behaviour of SunPy's visualisation 
routines intuitive to those who already understand the \texttt{matplotlib}
interface, as well as allowing the use of the standard 
\texttt{matplotlib} commands to manipulate the plot parameters (e.g., title, axes).
Data visualisation is provided by two functions: \texttt{peek()}, for quick 
plotting, and \texttt{plot()}, for plotting with more fine-grained control.

This section will give a brief overview of the \textit{current} functionality 
of each of the core SunPy data types.

\subsection{Map}\label{ssec:map}
The map data type stores 2D spatial data, such as images of the Sun and 
inner heliosphere. It provides: a wrapper around a \texttt{numpy} data array, 
the images associated spatial coordinates, and other metadata. The \texttt{Map} 
class provides methods for typical operations on 2D data, such as rotation and 
re-sampling, as well as visualisation.
The \texttt{Map} class also provides a convenient interface for loading data 
from a variety of sources, including from FITS files, the standard format for storing image data in solar physics and astrophysics community. 
An example of creating a \texttt{Map} object from a FITS file is shown in 
Listing~\ref{code:aia_1}.

The architecture of the map subpackage consists of a template map called
\texttt{GenericMap}, which is a subclass of \texttt{astropy.nddata.NDData}. 
\texttt{NDData} is a generic wrapper around a \texttt{numpy.ndarray} with a 
\texttt{meta} attribute to store metadata.
As \texttt{NDData} is currently still in development, \texttt{GenericMap} does 
not yet make full use of its capabilities, but this inheritance structure 
provides for future integration with \texttt{astropy}. In order to provide 
instrument- or detector-specific integration, \texttt{GenericMap} is designed
to be subclassed. Each subclass of \texttt{GenericMap} can register 
with the \texttt{Map} creation factory, which will then automatically return an instance
of the specific \texttt{GenericMap} subclass dependent upon the data provided. 
SunPy v0.5 has \texttt{GenericMap} specialisations for the following 
instruments: 

\begin{itemize}
\item \textit{Yohkoh}/SXT - SXT: Solar X-ray Telescope \citep{1991SoPh..136....1O, 1991SoPh..136...37T},
\item \textit{SOHO}/EIT and LASCO - SOHO: Solar and Heliospheric Observatory \citep{domingo1995}, EIT: Extreme Ultraviolet Telescope \citep{1995SoPh..162..291D} and LASCO: Large Angle Spectroscopic COronagraph \citep{1995SoPh..162..357B}
\item \textit{RHESSI} - RHESSI: Reuven Ramaty High Energy Solar Spectroscopic Imager \citep{2002SoPh..210....3L},
\item \textit{STEREO}/EUVI and COR1/2 - STEREO: Solar TErrestrial RElations Observatory \citep{2005AdSpR..36.1483K}, EUVI: Extreme Ultraviolet Imager \citep{2004SPIE.5171..111W} and COR1/2: CORonagraph 1/2 \citep{2002AdSpR..29.2017H}
\item \textit{Hinode}/XRT - XRT: X-Ray Telescope \citep{2007SoPh..243....3K, 2007SoPh..243...63G}.
\item \textit{PROBA2}/SWAP - PROBA2: PRojects for On Board Autonomy 2 \citep{2013SoPh..286....5S}, SWAP: Sun Watcher Active Pixel \citep{2013SoPh..286...43S}
\item \textit{SDO}/AIA and HMI - HMI: Helioseismic Magnetic Imager, \citep{2012SoPh..275..207S}
\item \textit{IRIS} - IRIS: Interface Region Imaging Spectrograph \citep{2011SPD....42.1512L} SJI (slit-jaw imager) frames.           
\end{itemize}
             
The \texttt{GenericMap} class stores all of the metadata retrieved from the header of
the image file in the \texttt{meta} attribute and provides convenience 
properties for commonly accessed metadata: e.g., \texttt{instrument}, 
\texttt{wavelength} or \texttt{coordinate\_system}.
These properties are dynamic mappings to the underlying metadata and all methods 
of the \textit{GenericMap} class modify the meta data where needed.
For example, if \verb|aiamap.meta[`instrume']| is modified then \verb|aiamap.instrument| 
will reflect this change.
Currently this is implemented by not preserving the keywords of the input data,
instead modifying meta data to a set of ``standard" keys supported by SunPy.
Listing \ref{code:aia_1} demonstrates the quick-look functionality of 
\texttt{Map}.

\begin{listing}[H]
\pythoncode{pycode_map1.txt}
\begin{center}
\includegraphics[width=0.8\columnwidth]{aia_map_example}
\end{center}
\caption{Example of the \texttt{AIAMap} specialisation of 
\texttt{GenericMap}. First, a map is created from a sample \textit{SDO}/AIA FITS file. In this case, a demonstration file contained within the SunPy repository is used. A cutout
of the full map is then created by specifying the desired solar-$x$ and solar-$y$ ranges of the plot in data coordinates (in this case, arcseconds), and then a quick-view plot is created with lines of heliographic longitude and latitude over-plotted.}
\label{code:aia_1}
\end{listing}

In addition to the data-type classes, the \texttt{map} subpackage provides two 
collection classes, \texttt{CompositeMap} and \texttt{MapCube}, for 
spatially and temporally aligned data respectively.
\texttt{CompositeMap} provides methods for overlaying spatially aligned 
data, with support for visualisation of images and contour lines overlaid 
upon each other.
\texttt{MapCube} provides methods for animation of its series of \texttt{Map} 
objects. Listings~\ref{code:compmap_1} and \ref{code:mapcube_1} show how to 
interact with these classes.

\begin{listing}[H]
\pythoncode{pycode_map2.txt}
\begin{center}
\includegraphics[width=0.8\columnwidth]{comp_map_example}
\end{center}
\caption{Example showing the functionality of \texttt{CompositeMap}, with RHESSI X-ray image data composited
on top of an \textit{SDO}/AIA 1600 $\AA$ image. The \texttt{CompositeMap} is plotted using the integration with the \texttt{matplotlib.pyplot} interface.}
\label{code:compmap_1}
\end{listing}

\begin{listing}[H]
\pythoncode{pycode_map3.txt}
\begin{center}
\includegraphics[width=0.8\columnwidth]{aia_cube_controls}
\end{center}
\caption{Example showing the creation of a \texttt{MapCube} from a list of AIA image files. The 
resultant plot makes use of \texttt{matplotlib}'s interactive widgets to allow scrolling 
through the \texttt{MapCube}.}
\label{code:mapcube_1}
\end{listing}

\subsection{Lightcurve}\label{ssec:lightcurve}

Time series data and their analyses are a fundamental part of solar
physics for which many data sources are available.
SunPy provides a \texttt{LightCurve} class
with a convenient and consistent interface for handling solar time-series
data.  The main engine behind the \texttt{LightCurve} class is
the {\texttt{pandas}} data analysis library.  
\texttt{LightCurve}'s \texttt{data} attribute is a \texttt{pandas.DataFrame} 
object. The \texttt{pandas} library contains a large amount
of functionality for manipulating and analysing time-series data,
making it an ideal basis for \texttt{LightCurve} \citep{mckinney2012}.  \texttt{LightCurve}
assumes that the input data are time-ordered list(s) of numbers, and each
list becomes a column in the \texttt{pandas} DataFrame object.

Currently, the \texttt{LightCurve} class is compatible with the
following data sources: the Geostationary Operational Environmental
Satellite (\textit{GOES}) X-ray Sensor (XRS), the \textit{Nobeyama
  Radioheliograph (NoRH)}, \textit{PROBA2} Large Yield Radiometer
(LYRA, \citealt{2013SoPh..286...21D}), \textit{RHESSI},
\textit{SDO} EUV Variability Experiment\footnote{Note that only the level ``OCS'' and average
  CSV files is currently implemented -- see
  \url{http://lasp.colorado.edu/home/eve/data/}} (EVE, \citealt{2012SoPh..275..115W}). 
\texttt{LightCurve}
also supports a number of solar summary indices - such as average
sunspot number - that are provided by the National Oceanic and
Atmospheric Administration (NOAA).  For each of these sources, a
subclass of the \texttt{LightCurve} object is initialised (e.g.,
\texttt{GOESLightCurve}) which inherits from \texttt{LightCurve}, but
allows instrument-specific functionality to be included.  Future
developments will introduce support for additional instruments and
data products, as well as implementing an interface similar to that of
\texttt{Map}.  Since there is no established standard as to how
time-series data should be stored and distributed, each SunPy
\texttt{LightCurve} object subclass provides the ability to download
its corresponding specific data format in its constructor and parse
that file type. A more general download interface is currently in development.

A \texttt{LightCurve} object may be created using a number of different methods. 
For example, a \texttt{LightCurve} may be created for a specific instrument given
an input time range. In Listing~\ref{code:goes_lc}, 
the \texttt{LightCurve} constructor searches a remote source for the GOES X-ray 
data specified by the time interval, downloads the required files, and 
subsequently creates and plots the object. Alternatively, if the data file 
already exists on the local system, the \texttt{LightCurve} object may be 
initialised using that file as input.

\begin{listing}[H]
\pythoncode{pycode_lightcurve.txt}
\begin{center}
\includegraphics[width=10cm]{goes_lightcurve.pdf}
\end{center}
\caption{Example retrieval of a GOES lightcurve
using a time range and the output of the 
\texttt{peek()} method. The maximum flux value in the \textit{GOES} 1.0--8.0$\AA$\ channel 
is then retrieved along with the location in time of the maximum.}
\label{code:goes_lc}
\end{listing}

\input{2-3-Spectra}
